\documentclass[12pt]{article} % Default font size is 12pt, it can be changed here
\usepackage{geometry} % Required to change the page size to A4
\usepackage{graphicx} % Required for including pictures
\usepackage{float} % Allows putting an [H] in \begin{figure} to specify the exact location of the figure
\usepackage{wrapfig} % Allows in-line images such as the example fish picture
\usepackage{amssymb}
\usepackage{url}
\geometry{a4paper} % Set the page size to be A4 as opposed to the default US Letter
\graphicspath{{graphs/}{img/}} % Specifies the directory where pictures are stored
\linespread{1.2} % Line spacing

\begin{document}

%----------------------------------------------------------------------------------------
%	TITLE PAGE
%----------------------------------------------------------------------------------------
\begin{titlepage}

\newcommand{\HRule}{\rule{\linewidth}{0.5mm}} % Defines a new command for the horizontal lines, change thickness here

\center % Center everything on the page

\textsc{\LARGE Universit\'a di Trento}\\[0.8cm] % Name of your university/college
\textsc{\Large Corso HCI A.A. 2014-2015}\\[0.8cm] % Major heading such as course name
\textsc{\large Partecipatory Development}\\[1.5cm] % Minor heading such as course title

\HRule \\[0.8cm]
{ \huge \bfseries Gestione delle labels di Github}\\[0.4cm] % Title of your document
\HRule \\[2cm]

\begin{minipage}{0.4\textwidth}
\begin{flushleft} \large
\begin{tabular}{ll}
Bortoli \textsc{Gianluca} & \makebox[2cm][r]{159993} \\
Brugnara \textsc{Martin} & \makebox[2cm][r]{157791} \\
Dellera \textsc{Andrea} & \makebox[2cm][r]{158365} \\
Hoxha \textsc{Fatbardha} & \makebox[2cm][r]{161003}
\end{tabular}
\end{flushleft}
\end{minipage}

\vfill % Fill the rest of the page with whitespace
{\large \today}\\[3cm] % Date, change the \today to a set date if you want to be precise
\end{titlepage}

%----------------------------------------------------------------------------------------
%	TABLE OF CONTENTS
%----------------------------------------------------------------------------------------
\tableofcontents % Include a table of contents

\newpage % Begins the essay on a new page instead of on the same page as the table of contents 

\section{Introduzione}
Il primo obiettivo di questo progetto consiste nell'analizzare è quali siano le principali barriere nella partecipazione ai progetti open source, che oggigiorno sono diventati sempre più numerosi e di una certa rilevanza (addirittura a livello mondiale in alcuni casi).\\
Il nostro interesse è ricaduto proprio in queste problematiche che affliggono questo tipo di progetti, dal momento che sono state ravvisate anche in prima persona durante il corso dei nostri studi.\\
Questo approfondimento deriva in parte anche dalla nostra propensione ed interesse per l'utilizzo di software open source durante la nostra carriera universitaria e lavorativa.\\
Lo scopo finale è quello di trovare e formulare una possibile soluzione al problema della gestione delle etichette su Github (la piattaforma principe per lo sviluppo open source), dal momento che essa è attualmente molto confusionaria e poco ben gestita, soprattutto in progetti di una certa complessità e grandezza.
\newpage

\section{Interviste}
Il primo passo per l'individuazione delle principali problematiche legate all'ambito della partecipazione ai progetti open source è stata fatta tramite delle interviste. Questo tipo di ricerca si presta molto alla valutazione di barriere di questo tipo, dal momento che l'intervistato non si sente limitato nell'esprimersi (come potrebbe risultare da un questionario a risposta multipla), ma al contempo non ha la percezione di dilungarsi troppo (come invece potrebbe accadere se viene presentato un questionario a domanda aperta) che potrebbe indurlo a non scrivere tutto ciò che pensa.\\
Con il tramite dell'intervista "faccia a faccia" questi inconvenienti vengono meno: la persona si sente più libera di discutere con l'intervistatore che pone le domande e non blocca l'intervistato mentre sta rispondendo, non interrompendo così il flusso delle idee (che è la parte essenziale e di reale interesse dell'intera intervista).\\
\subsection{Traccia}
La traccia \footnote{\url{http://disi.unitn.it/~deangeli/homepage/lib/exe/fetch.php?media=teaching:hci:hci2014_2015:intervista_pd_motivazioni.pdf}} delle domande da porre all'intervistato ci è stata fornita direttamente dalla dottoressa Bordin, poiché tale argomento non era ancora stato trattato a lezione nel momento in cui abbiamo svolto questa parte del progetto.\\
Essa prevedeva delle domande mirate principalmente a capire:
\begin{itemize}
\item se e quali software open source vengono utilizzati
\item se l'intervistato partecipi/abbia partecipato attivamente a tali progetti
\item quali siano le motivazioni che lo hanno portato a farlo oppure no
\item cosa significhi \emph{partecipare} ad un progetto open source
\end{itemize}
\subsection{Analisi dei risultati}
Il campione su cui è stata effettuata l'intervista non è molto eterogeneo (come è possibile vedere dalla Figura \ref{fig:distribuzione}), dal momento che è stato più immediato intervistare delle persone all'interno del nostro corso di laurea piuttosto che di altri atenei o dei lavoratori.

\begin{figure}[H] 
\center{\includegraphics[width=0.8\linewidth]{interviewed_distribution.png}}
\caption{Distribuzione della professione delle persone intervistate.}
\label{fig:distribuzione}
\end{figure}

Abbiamo potuto notare come gli unici due intervistati che hanno contribuito attivamente a progetti open source provenissero esclusivamente da un ambito informatico. Inoltre, all'interno del gruppo stesso ben pochi lo hanno fatto, come è possibile notare dalla Figura \ref{fig:distribuzioneInformatica}.\\
Ciò ci permette di creare un profilo di utenza specifica (una \emph{personas}) in grado di prendere parte a questa tipologia di progetti. 

\begin{figure}[H] 
\center{\includegraphics[width=0.8\linewidth]{interviewed_participation.png}}
\caption{Persone che hanno attivamente partecipato a progetti open source tra gli studenti di informatica.}
\label{fig:distribuzioneInformatica}
\end{figure}

Da ciò si evince che l'ambito e la possibilità di parteciparvi è molto ristretto e ciò ci permette di delineare una \emph{personas} con delle caratteristiche ben definite, quali:
\begin{itemize}
\item consistente background informatico
\item interesse verso il campo specifico del progetto considerato
\item tempo e impegno da dedicarvi
\end{itemize}

In aggiunta abbiamo notato come lo \emph{scenario} accademico incentivi e contribuisca ad utilizzare specifici software open source e, di conseguenza, aumenti la probabilità che uno studente si interessi e ne possa far parte. 
%-------
% descrizione di cosa sia una personas?
%-------
\newpage

\section{Benchmarking}
Durante questa fase abbiamo scelto di concentrarci su tre progetti open source: NeoVim, CyanogenMod e OpenOffice.\\
Dopo averli presentati e descritti durante un workshop, abbiamo analizzato i pro ed i contro di ciascuno, da cui è emerso che molti utenti utilizzano spesso questi software sia in ambito personale che lavorativo.\\
Quelli da noi considerati sono gratuiti \footnote{Non bisogna confondere il concetto di open source con quello di freeware: che un software sia open source non implica necessariamente che sia anche distribuito gratuitamente (ad esempio RedHat)}, di facile utilizzo e mantenuti costantemente aggiornati dalla comunità di sviluppo. In più le funzionalità offerte sono molto simili a quelle dei rispettivi software equivalenti, se non addirittura le medesime.\\
Al contrario, essi non vengono largamente utilizzati a causa della scarsa conoscenza del prodotto, dal momento che vengono scarsamente pubblicizzati soprattutto a livello delle istituzioni pubbliche, e per la loro usabilità sulla quale spesso non viene fatto uno studio molto approfondito.\\
\subsection{Contribuire ad un progetto}
Dalle interviste è emerso che il significato dato al termine \emph{contribuire} è quello di mettere a disposizione le proprie competenze per il miglioramento di un software e/o la risoluzione di eventuali problemi ad esso legati.\\
Le motivazioni che spingono a contribuire sono molte: dalla semplice soddisfazione per aver risolto un bug, all'arricchimento personale e l'esperienza acquisita che ne consegue, passando per la necessità di dover implementare una funzionalità che risolva un problema che si ha nel proprio lavoro.\\
Al contrario, altrettanti sono i motivi che spingono le persone a non contribuire, tra i quali la mancanza di tempo e di competenze e soprattutto la qualità e disponibilità di informazioni riguardanti un progetto specifico. Quest'ultimo a nostro avviso risulta essere il più rilevante, dal momento che un utente (con magari delle capacità per farlo) si trova a non poter contribuire a causa della mancanza di linee guida su come si possa farlo. Ciò potrebbe sembrare banale a primo impatto, ma è una costante in tutti i progetti da noi presi in considerazione.
\newpage

\section{Design library: NeoVim}
Nella nostra \emph{design library} ci siamo concentrati sul progetto di NeoVim per poter individuare sia degli esempi di cattiva e che di buona progettazione da cui prendere spunto.\\
\subsection{Buoni esempi}

\begin{figure}[H] 
\center{\includegraphics[width=0.8\linewidth]{buonesempio1.png}}
\caption{La homepage del sito di NeoVim indica chiaramente una sezione per chi volesse sviluppare.}
\label{fig:buonesempio1}
\end{figure}

Si può già notare come il gruppo principale di sviluppo del progetto preso in considerazione indichi chiaramente già nella pagina principale una sezione per chi volesse sviluppare.

\subsection{Cattivi esempi}

\end{document}

%----------------
% miglioramenti: 
%				campione più eterogeneo e più vasto per le interviste
%---------------

